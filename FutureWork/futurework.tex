\chapter{Future work}
% The future plan is to transition from the current experiment, which involves working on the ARM architecture on the ARM Morello board. 
% The current limitation is that all memory reads must go through the TLB for translations. The future plan involves storing 
% the offset directly on the pointer and using the bounds in CHERI to enable block-based allocator behavior. This phase of the 
% experiment is intended to be conducted on the RISC-V implementation of CHERI, known as Tooba. 

% In the RISC-V implementation, the hardware Verilog design will be modified to allow bypassing the TLB. Once this concept is 
% integrated into the RISC-V Verilog implementation, the OS layer will be changed to a single-address-space operating system, 
% where there is no distinction between user space and kernel space. In this implementation, the kernel allocator will be the 
% same as the user space allocator since both can utilize a single contiguous chunk of memory.
The current experimental setup on the ARM Morello board is constrained by the requirement that all memory reads must 
pass through the Translation Lookaside Buffer (TLB) for address translation. This necessitates frequent TLB lookups, potentially 
leading to performance bottlenecks. The planned future work aims to address this by leveraging CHERI 
(Capability Hardware Enhanced RISC Instructions) extensions on the RISC-V architecture, specifically using the 
Tooba implementation\cite{noauthor_ctsrd-cheride10pro-cheri-bgas_2024}.

\subsubsection{Storing Offsets Directly on Pointers}
In the current ARM Morello setup, address translations rely on the TLB.
The future approach on RISC-V Tooba involves storing the offset directly within the pointer. This is possible due to CHERI's capability model, which supports fine-grained memory protection and can encode bounds within pointers.
Utilizing Bounds in CHERI for Block-Based Allocation:

CHERI capabilities allow pointers to carry metadata about memory bounds, providing hardware-enforced memory safety.
By encoding the offset and bounds within the pointer, the system can directly access memory without needing intermediate translations via the TLB.
This enables the implementation of a block-based allocator that can efficiently manage memory allocations and deallocations within defined bounds.
Bypassing the TLB in RISC-V Tooba.
\subsubsection{Hardware Modifications:}
The Bluespec design of the RISC-V processor will be modified to allow certain memory operations to bypass the TLB. This means that when a pointer with encoded offset and bounds is used, the system can directly compute the physical address from the capability information.
This modification reduces the dependency on the TLB, decreasing latency and improving performance, especially for frequent memory operations.
Transition to a Single-Address-Space Operating System (SASOS)\cite{esswood_cherios_nodate}.
\subsubsection{Concept of SASOS:}
In traditional operating systems, there is a clear separation between user space and kernel space. This separation is enforced by memory protection mechanisms and address translation through the TLB.
In a Single-Address-Space Operating System, this distinction is removed. Both user applications and the kernel share the same contiguous address space.
\subsubsection{Advantages of SASOS with CHERI:}
% Rewrite this bit
\begin{itemize}
  \item Simplified Memory Management : Without the need to switch between user and kernel spaces, memory management becomes simpler and more efficient.
The kernel allocator can be the same as the user space allocator, operating on a single, contiguous chunk of memory.
  \item Unified Allocator: The unified memory allocator can efficiently manage memory for both kernel and user applications, leveraging CHERI's capability-based protection to prevent unauthorized access.
This reduces overhead and potential fragmentation issues associated with maintaining separate memory spaces.
\end{itemize}