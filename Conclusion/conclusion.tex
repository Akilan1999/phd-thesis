%!TEX root = ../thesis.tex
%*******************************************************************************
%*********************************** Conclusion *****************************
%*******************************************************************************

\chapter{Conclusion}  %Title of the Conclusion
The report addresses the growing disparity between application workloads and the capacity of Translation Lookaside Buffers (TLBs). 
To mitigate this gap, it proposes leveraging physically contiguous memory to optimize TLB utilization. Additionally, 
the report explores advancements in system security, particularly through the Capability Hardware Enhanced RISC Instructions (CHERI) 
architecture. CHERI's capability-based addressing enhances system security by associating capabilities with memory pointers, 
restricting access to memory regions, and thus protecting against various security threats. Importantly, these mechanisms 
can also improve the efficiency of memory allocators by managing memory resources while ensuring robust security measures.
\newline

The report highlights the constant pursuit of optimal performance in computing, emphasizing the importance of 
efficient memory management. TLBs are crucial in expediting memory access by storing recently accessed memory translations. 
However, as applications grow in size and complexity, TLB capacity often becomes a bottleneck. One innovative solution 
is the use of huge pages, which allocate memory in larger chunks, thereby reducing the number of TLB entries required 
and potentially enhancing overall system performance. Advancements in hardware-level security, such as CHERI's 
capability-based addressing, offer additional performance enhancement opportunities by tightly controlling memory 
access and accelerating memory management operations. Integrating huge pages into memory management strategies 
alongside CHERI's capability-based addressing can optimize TLB utilization and leverage security features for 
significant performance improvements.
\newline
% The future work section outlines the planned research timeline, focusing on the development and evaluation of FAT-pointer-based 
% range addresses. Key milestones include the initial development phase in July 2024, followed by integration with the RISC-V architecture 
% from August to September 2024. Detailed testing and evaluation are scheduled from October 2024 to February 2025, with an extension 
% of the implementation to uni-kernels planned from March to May 2025. Finalization and optimization of the approach are expected 
% from June to September 2025, culminating in a comprehensive evaluation and documentation of the results from January to September 2026.
% \newline

This report aims to demonstrate how leveraging physically contiguous memory and advanced security architectures like CHERI can 
enhance memory management efficiency while ensuring robust security measures. These advancements ultimately contribute to 
improved system performance, addressing the challenges posed by the increasing complexity and size of modern application workloads.