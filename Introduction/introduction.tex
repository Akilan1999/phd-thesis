%!TEX root = ../thesis.tex
%*******************************************************************************
%*********************************** Introduction *****************************
%*******************************************************************************

\chapter{Introduction}  %Title of the Introduction
Modern software practices aim for high performance and better security
and scalability at the time. There have been few areas of research 
which have been of great interest in the recent research in support 
of providing a high performant system with better security as well.
Few of ways this has been achieved is by trimming down the size of 
the operating systems, by building a dedicated hardware to manage 
help manage and execute security policies and the third approach 
which is a paradigm to support running a single program across 
various types of cores which could be completely different 
architectures.

\subsubsection{Trimmed down OS}
The approach of interest in context of this research would 
be Uni-kernels. Uni-kernels is specialized single address 
spaced OS image constructed by using the Library operating 
system model. A library operating system is the standard services 
provided by a typical operating system such as networking provided 
in a form of libraries which is then constructed with the application 
either at compile time or as a separate process after compile time.

\subsubsection{Dedicated hardware to manage and execute security
policies}
In context of the following TAG based architectures would be 
of interest. Tagged based architecture can be classified as 
hardware security primitives which consists of data and code with tags. 
Tags function as security metadata mostly about memory. This is
created before run-time. During run-time the hardware 
would enforce the following policies which in return 
provides security guarantees. 

\subsubsection{Running a single program across cores of 
various architectures}
In context of the following the Multi kernel approach would 
be of interest. Multi-kernel approach treats a multi-core machine 
as a network of independent cores. This means a program would interact
with multiple cores the same way as it would interact with a distributed systems 
using message passing as an example. 

\subsection{Discussion}
Based on the following 3 subsections the aim of the 
following PhD would be combine all the 3 approaches into a 
single system which would allow programs to be more scalable, 
provide the possibility to run certain parts of a program on secure 
hardware and improve the performance by using a slim down kernel. 
The literature review section covers the following in depth. 

\subsection{Organization of this report}
The following report is organized into 6 sections:
\begin{itemize}
    \item Introduction: Covers the introduction of the following report.
    \item Research Questions: Covers the identified research questions 
    based on activities conducted in year 1.
    \item Literature Review: Covers a full survey of implementations/ papers 
    on Uni-kernels, Multi-kernels and TAG based architectures.
    \item Year 1 Activity: Covers the activities conducted during the year 1 
    of the following research.
    \item Research Timeline: Provides a proposed timeline of activities to be conducted 
    during the following research.
    \item Conclusion: Provides a summary of the following report.
\end{itemize}


