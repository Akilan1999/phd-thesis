]%!TEX root = ../thesis.tex
%*******************************************************************************
%*********************************** Literature Review *****************************
%*******************************************************************************

\chapter{Literature Review}  %Title of the Literature Review

\ifpdf
    \graphicspath{{LiteratureReview/Figs/Raster/}{LiteratureReview/Figs/PDF/}{LiteratureReview/Figs/}}
\else
    \graphicspath{{LiteratureReview/Figs/Vector/}{LiteratureReview/Figs/}}
\fi

%********************************** %Introduction for literature review **************************************

The literature review is split into 3 sections. The first section talks about the papers surveyed 
for Unikernels and the 2nd section talks about papers surveyed for TAG based architectures and 
the third sections talks about the possible incentives of combining them both which helps 
answer the research questions stated (TODO: Add reference to research question section). 

\section[TAG based architecture survey]{TAG based architecture survey}   
The following was a survey conducted on exisisting TAG based implementations and the 
recent survey based on TAG based architectures (//TODO add survey reference) published
in 2022 was a good staring point to understand about various implementations of TAG
based archtectures with the high level metrits and limitations. The following section 
provides our own version of the Survey to help decide the best implementations 
to answer the research questions (//TODO reference research questions chapter). 
 
According to the TAG based architecture survey (//TODO add survey reference) there are 37 published
efforts on TAG based architectures over the past decade and 20 published efforts preceding that. 

\subsection{Timder V}
 It is a tagged memory architecture for flexible and efficient isolation of code and data on 
 small embedded systems. The TAG isolation is augmented with a memory protection unit to isolate 
 induvidual processes. Timber V is compatible with exsisting code. The contributions of the paper 
 are: 
 \begin{itemize}
  \item Efficient tagged memory architecture for isolated execution on low-end processors. 
  \item Concept introducted called stack interleaving that allows efficient and dynamic memory management. 
  \item Lightweight shared memory between enclaves. 
  \item Efficient shared MPU (i.e Memory Protection Unit) design. 
 \end{itemize}
 

%~\ref{weiser_timber-v_2019}
%           FORMAT     
% TODO 
%  - Get bibtex file for references for TAG based architecture 
%  - Start paragraph 
%  - Benchmarks and test performaned  
	
\subsection{ARM MTE}
The ARMv8.5-Memory Tagging Extension (MTE) aims to increase the memory safety written for 
unsafe languages without requiring source code changes and in certain cases without 
recompilation. It generally foicuses on the bounds checking use case, Though it 
provides limited tags which means it can only provide probablilistic overflow detection. 
It is one of the latest commercial incarnations of memory-safety-focused tagged architectures.   

\subsection{D-RI5CY}
It provides a design a design and implementation of a hardware dynamic information flow 
tracking (DIFT) architecture for RISC-V processor cores. The paper presents a low 
overhead implementation of DIFT that is specialized for low-end embedded systems
for IOT applications. The following are high level contributions:
\begin{itemize}
  \item Design f D-RI5CY, A DIFT-protected implementation of the RI5CY processor core. 
        The paper implements the modification of the DIFT TAG propogation and TAG checking
        mechanism in a way that is transparent to the execution of the regular instructions. 
  \item Concept introducted called stack interleaving that allows efficient and dynamic memory management.
  \item Lightweight shared memory between enclaves.
  \item Efficient shared MPU (i.e Memory Protection Unit) design.
\end{itemize}


\subsection{TMDFI}


\subsection{HyperFlow} 
It is a design and security implementation that offers security assurance because it is implemented 
using a security-typed hardware description language. It allows complex information flow policies to 
be configured at run time. The paper introduces ChiselFlow, a new secure hardware description language. 
The contribution of the paper includes: 
\begin{itemize}
  \item Processor architecture and implementation designed for timing-safe information flow security. 
  \item Complere RISC-V intruction set extended with intructions for information flow control. 
  \item Verified at design time with a hardware description language. 
  \item Novel representations of lattices that can be implemented in hardware efficiently. 
\end{itemize}
Hyperlow implements a nonmalleable IFC policy using tags.
To eliminate timing side channels, the processor tracks the tag of the currently executing code and lushes caches,
TLB, branch predictor, and other microarchitectural state on changes in the conidentiality or integrity tag of the
running code. The modifications to avoid timing side channels seem more extensive than those to add tags. The
authors report overheads in cycles per instruction of between 1% and 69%, largely due to padding the multiply
operation to the worst-case number of cycles.

\subsection{SDMP}
This paper focuses on designing metadata tag based stack-protection security policies for general purpose tagged
architecture. The policies specifically
exploit the natural locality of dynamic program call graphs to
achieve cacheability of the metadata rules that they require.
The simple Return Address Protection policy has a performance
overhead of 1.2\% but just protects return addresses.
The two richer policies present, Static Authorities and Depth Isola-
tion, provide object-level protection for all stack objects. When
enforcing memory safety, The Static Authorities policy has a
performance overhead of 5.7\% and our Depth Isolation policy
has a performance overhead of 4.5\%.
The contribution of the paper includes:
\begin{itemize}
  \item The formulation of a range of stack protection policies
within the SDMP model.
  \item Three optimizations for our stack policies: Lazy Tagging,
Lazy Clearing and Cache Line Tagging.
  \item The performance modeling results of our policies on
a standard benchmark set, including the impact of our
proposed optimizations.
\end{itemize}  

\subsection{Typed Architecture}
This paper introduces Typed
Architectures, a high-efficiency, low-cost execution sub-
strate for dynamic scripting languages, where each data
variable retains high-level type information at an ISA level.
Typed Architectures calculate and check the dynamic type
of each variable implicitly in hardware, rather than explicitly
in software. Typed Architectures provide
hardware support for flexible yet efficient type tag extraction
and insertion, capturing common data layout patterns of tag-
value pairs. The evaluation using a fully synthesizable RISC-
V RTL design on FPGA shows that Typed Architectures
achieve mean speedups of 11.2\% and 9.9\% with
imum speedups of 32.6\% and 43.5\% for two production-
grade scripting engines for JavaScript and Lua. 
The contribution of the paper includes:
\begin{itemize}
  \item ISA extension to efficiently manage
type tags in hardware, which can be flexibly applied to
multiple scripting languages and engines.
  \item Design and implement the Typed Architecture pipeline,
which effectively reduces the overhead of dynamic type
checking at low hardware cost.
  \item Prototype the proposed processor architecture using 
a fully synthesizable RTL model to execute two
production-grade scripting engines with large inputs on
FPGA (executing over 274 billion instructions in total)
and provide a more accurate estimate of area and power
using a TSMC 40nm standard cell library.
\end{itemize}

\subsection{Dover}


\subsection{Shakti-T}

\subsection{HDFI}

\subsection{lowRISC}

\subsection{Taxi}

\subsection{Pump}

\subsection{CHERI}

\subsection{SPARC M7/M8 SSM}

\subsection{Low-Fat Pointers}

\subsection{SAFE}

\subsection{DataSafe}

\subsection{Harmoni}

\subsection{Shioya, et al.}

\subsection{SIFT}

\subsection{FlexCore}

\subsection{Execution Leases}

\subsection{GLIFT}

\subsection{TIARA}

\subsection{DIFT Coprocessor}

\subsection{HardBound}

\subsection{Loki}

\subsection{FLexiTaint}

\subsection{SECTAG}

\subsection{Raksha}

\subsection{SecureBit}

\subsection{Minos}

\subsection{DIFT}

\subsection{RIFLE}

\subsection{AEGIS}

\subsection{Mondriaan}

\subsection{Aries}

\subsection{XOM}

\subsection{M-Machine}

\subsection{KCM}

\subsection{SPUR}

\subsection{Lisp Machine}

\subsection{HEP}

\subsection{Burroughs}






































